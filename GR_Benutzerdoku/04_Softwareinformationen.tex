%Softwareinformationen

\ihead{\headmark}
\ohead{Lars Strölin, Michael Geigges, Ilja Kononenko}
\automark{section}
\cfoot{\pagemark}


\section{Softwareinformationen}

\subsection{Installation}
Um das Programm auf einem PC zu benutzen benötigt man unsere selbst programmierte JAR-Datei und eine JAVA Version mit der Versionsnummer 1.8.0 oder höher. Nach einem Doppelklick auf die JAR-Datei öffnet sich das Programm und alles ist sofort voll funktionsfähig und benötigt keine weiteren Installationen. 



\subsection{Einführung}
Dieses PC-Programm ermöglicht einem oder mehreren Benutzern durch eine sehr übersichtliche und kompakte Oberfläche die Eingabe einer Ganzrationalen Funktion, die Anzeige der Funktion auf einem Koordinatensystem, sowie jeweils die Ableitung, die Berechnung der Extrempunkten und der Nullstellen. 
\newline
Zusätzlich können mit einem eingebauten Skalierungs-Menü die Skalierungswerte, also sprich X-Min, X-Max, Y-Min und Y-Max eingestellt werden, sodass man seine Funktion auf jeden Fall komplett sehen kann, oder bestimmte Punkte ganz gezielt durch den Zoom betrachten kann.
Eine weitere Funktion, die das Programm besitzt, ist das Anzeigen des Y-Wertes nach Eingabe einer Funktion mit einem X-Wert.



\newpage