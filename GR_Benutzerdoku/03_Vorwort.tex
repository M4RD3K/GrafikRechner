%Vorwort

\ihead{\headmark}
\ohead{Lars Strölin, Michael Geigges, Ilja Kononenko}
\automark{section}
\cfoot{\pagemark}


\section{Vorwort}
Zu Beginn mussten wir uns zu dritt erst einmal zusammensetzen um zu besprechen was genau wir in unserem IT-Projekt erschaffen wollen. Die erste Frage war Android oder Java? Da wir alle mit Java schon mehr Erfahrung hatten und außerdem ein Programm für den PC erstellen wollten, war dies eine recht einfache Sache. Jetzt mussten wir nur noch wissen was für ein Programm wir erstellen und was dies können soll. 
\newline
Wir alle waren schon in der Lage mit Java Summen und andere mathematischen Formeln zu berechnen und hatten bereits einige Übungen, wie zum Beispiel da, wo wir unseren eigenen kleinen Taschenrechner mit den gewöhnlichen Aktionen Addition, Subtraktion, Multiplikation und Division programmiert haben. Dieses Mal sollte es aber etwas umfangreicheres werden und da wir in dem Fach "Mathe" gerade Funktionen behandelt haben, war unser erster Gedanke ein Programm welches bei Eingabe einer Funktion die entsprechende Grafik dazu anzeigt.
\newline
\newline
Diese Benutzerdokumentation beschreibt systematisch die Benutzung des Programms Grafikrechner. Diese sind in zwei Hauptthemen: Systeminformationen und Funktion gegliedert bzw. aufgeteilt.
Innerhalb der jeweiligen Themen werden mithilfe von Bildern erklärt, wie das Programm verwendet werden kann. Um die Bedienung der Programme so schnell wie möglich zu gewährleisten, sind die Beschreibungen deswegen so kurz wie möglich und kompakt gehalten worden. 




\newpage