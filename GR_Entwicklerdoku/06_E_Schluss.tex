%Vorwort

\ihead{\headmark}
\ohead{Lars Strölin, Michael Geigges, Ilja Kononenko}
\automark{section}
\cfoot{\pagemark}


\section{Fazit}

Im Großen und Ganzen ist es uns gelungen ein Programm zu entwickeln, welches durch Eingabe einer Funktion diese in einem Schaubild korrekt darstellt und dazu auch besondere Werte, wie z.B. Nullstellen und Extrempunkte ausrechnet. Es war aber dennoch für uns sehr schwierig diese Aufgabe zu realisieren, da wir uns erst Gedanken machen mussten, wie wir die Erstellung unseres Programms angehen und vor allem wie wir die mathematischen Formeln programmieren.\\

Die größten Schwierigkeiten und Problematiken waren deshalb die Codes für die Ausrechnung der besonderen Werte für unser Programm zu entwickeln. 
Doch genau aus diesem Grund sind wir zufrieden es geschafft zu haben und können jetzt im Programm Verbesserungen durchführen. Die Oberfläche muss auf jeden Fall noch benutzerfreundlicher gestaltet und zum Teil automatisiert werden.\\

Außerdem hatten wir uns eigentlich vorgenommen eine Funktion zu entwickeln, in der man bestimmte Funktionen abspeichern und wieder aufrufen kann, da uns aber die Zeit dazu gefehlt hat und diese Funktionen optional waren, fielen sie für die Abgabe des jetzigen Programms raus.